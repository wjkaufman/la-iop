\documentclass[10pt]{article}

\usepackage[top=0in, bottom=0in,left=0in, right=0in, paperwidth=10in, paperheight=5in]{geometry}

%\usepackage{pgfpages}
%\pgfpagesuselayout{2 on 1}

\begin{document}
\title{\textit{King Henry IV Part 1}: A Historical Analysis}
\author{Will Kaufman --- Mr. Bursiek --- 20 February 2015}
\date{}
\maketitle

\pagenumbering{gobble}
\section*{Thesis}
Shakespeare takes various artistic liberties within King Henry IV Part 1 to present the royalty in a more positive light than is historically accurate and condemn certain religious groups at the time.  Portraying the royalty positively and condemning Lollardy allowed the play to become more popular, and achieve more theatrical success.

\section*{Differences in the Play}
\subsection*{Portrayal of Prince Henry}
Prince Henry grew up close with his father, but eventually began to oppose his father's will.  Shakespeare presents Prince Henry as troublesome and conflicted with his father, but brings the characters together through the course of the play to present a united royalty.  Shakespeare's interpretation is opposite of what actually happened in history.
\textit{Relevant passages: } \textbf{I.ii.145, III.ii.125, V.iv.46}

\subsection*{Portrayal of John Falstaff}
Sir John Oldcastle, the knight that Falstaff is based on, was a distinguished soldier under King Henry IV, but was killed for following the Lollard faith.  Falstaff is depicted as a contemptible character to negatively portray Oldcastle and Lollardy in general.
\textit{Relevant passages: } \textbf{II.iv.370}

\noindent\rule{\textwidth}{1pt}
\end{document}
